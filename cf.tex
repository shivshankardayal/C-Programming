\chapter{Flow Control}
There are four things you will learn in this chapter. Switching the path of 
execution in program depending upon program variables or states using control 
statements, repeating a set of instructions using loops, bypassing certain set
of instructions in a loop and jump around. Collectively, these elements of C
allow or enable you to take driver's seat over the control over a C program.
You will spend much of your programming time even in future using these basic
elements. It is critical to understand the these topics well as these are basic
pillars over which rest of chapters will build upon.

Let us start with if-else statement \S(6.8.4.1) which is part of selection
statements \S(6.8.4).

\section{if else-if else Statements}
An if statement can be broken into three distinct part. It starts with a
mandatory single `if` clause which tests an expression and if that expression
evaluates to boolean true then an associated block of code is executed. The
`if` part may be followed by zero or more `else if` statements which also test
an expression and it can have an associated block of code as well. Finally it
can have an `else` statement which is optional and does not have any expression
to test against. Rather if all above statements did not match their expressions
then else block's code will be executed. Note that among all blocks of code of
`if, else if` and `else` only one block of code will execute and rest will
not.

Let us see a small program to see these in action:

\begin{Verbatim}[frame=single]
#include <stdio.h>

int main()
{
  int i = 0, j= 0;
    
  printf("Please enter two integers i and j:\n");
  scanf("%d%d", &i , &j);
  
  if(i==4)
    printf("you entered 4 for i.\n");

  if(i==7)
  {
    printf("you entered 7 for i.\n");
    printf("I am happy for you.\n");
  }
  else
  {
    printf("You did not enter 7 for i.\n");
  }
  
  if(i==7)
  {
    printf("you entered 7 for i.\n");
    printf("I am happy for you.\n");
  }
  else if(j==8)
    printf("You entered 8 for i.\n");
  
  if(i==7)
    printf("you entered my lucky number.\n");
  else if((i==7) &&(j==8))
    printf("May god bless you!\n");
  else
    printf("You entered bad number.\n");
  
  return 0;
}
\end{Verbatim}

and the output is:

\begin{Verbatim}[frame=single]
Please enter two integers i and j:
4
6
you entered 4 for i.
You did not enter 7 for i.
You entered bad number.
\end{Verbatim}

As you can see from first if sttatement that if you enter the value of `i` as 4
then the `printf` will be executed and you will be able to see it. Note that if
there are multiple lines below if which you want to execute then you must put
them in a block using curly braces. If you just want to execute one line then
these curly braces are optional. Note that how you must use curly braces if you
have more than one line and you want to execute that block. Also, see the
syntax for `else` and `else if`. One if-else can be nested inside
another for example see the following code:

\begin{Verbatim}[frame=single]
#include <stdio.h>
#include <string.h>
 
int main()
{
  char fName[128]={0}, lName[128]={0};
 
  printf("Enter your first name and last name in that order:\n");
  gets(fName);
  gets(lName);
 
  if(strcmp(fName, "Shiv") == 0)
  {
    if(strcmp(lName, "Dayal") == 0)
      printf("Your name is Shiv Dayal.\n");
  }
  else
  {
    printf("Your name is %s %s.\n", fName, lName);
  }
 
  return 0;
}
\end{Verbatim}

and the output is:

\begin{Verbatim}[frame=single]
Enter your first name and last name in that order:
Shiv
Dayal
Your name is Shiv Dayal.
\end{Verbatim}

another run when first `if` fails:

\begin{Verbatim}[frame=single]
Enter your first name and last name in that order:
Richard
Stallman
Your name is Richard Stallman.
\end{Verbatim}

when first `if` matches but second `if` does not and this we have no output:

\begin{Verbatim}[frame=single]
Enter your first name and last name in that order:
Shiv
Stallman
\end{Verbatim}

Note the usage of nested if-else. Also, note how `strcmp` has been used to
compare two strings and `gets` to read the input. `strcmp` takes two character
strings as argument and returns 0 if they are equal. It returns non-zero values
depending on whether one string is lexically greater than the other or not. But
for now equality is enough for us. `gets` is dangerous but it is simple that is
why has been used here. It is easy to overflow the buffer of `gets` argument.
