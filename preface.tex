\chapter{Preface}
I welcome you to this pdf version of 
\url{https://10hash.com/books/c/} which is a complete rewrite. That 
was my first attempt to write at such a large scale and therefore there were 
many deficiencies in that version which I hope have been countered and
corrected in this version. In the HTML version I had put a lot of emphasis on 
specification. However, the feedback which I got from some of my readers is 
that it has become complicated and is not suitable for beginners. After 
thinking over it I found that it is the order of chpaters which needs to be 
corrected and a lot more examples and exercises need to be there.

In this rewrite I will arrange the chapters in a different way. The empahsis on 
specification will be treated in the last part of book.

\section*{What this book contains?}
This book is about C programming language. In this book I have explained the
syntax of C programming language. But it is not only for learning about
language but how to learn to program using C programming language is also
there. In this book we will follow latest ISO specification of C which is
popularly known as C11.

\section*{Who should read this book?}
People interested in learning C should read this. Anybody can read this book
who has little background on Mathematics. I would say high school education is
sufficient for reading it. So this book is for both beginners and experts
alike. For experts advanced concepts have been presented and also a reference
to standard library has been given along with its usage.

\section*{How to read this book?}
Well, learning programming is like learning a new language and then solving
problems is like Mathematics. So the latter is more important as unless you can
solve the problem on pen and paper, you cannot solve the problems using
C. However, the focus of this book is to explain the features and syntax of C
programming language not on how to solve the problem. To get a good grasp
on the contents of the book one should read the theory first then look at the
examples given and try to understand and run them and in the end attempt
the problems. If you cannot solve the problems then lookup the solution and
try to understand. In case of any problem please log on to my personal
question/answer website \url{http://kunjika.libreprogramming.org/} to get
more help.

\section*{Acknowledgements}
I am in great debt of my family and free software community because both of
these groups have been integral part of my life. Family has prvided direct
support while free software community has provided the freedom and freed me
from the slavery which comes as a package with commercial software. I am
especially grateful to my wife, son and parents because it is their time which
I have borrowed to put in the book. To pay my thanks from free software
community  I will take one name and that is Richard Stallman who started all
this  and is still fighting this never-ending war. When I was doing the Algebra
book then I realized how difficult it is to put Math on web in HTML format and
why Donald Knuth wrote \TeX{}. Also, \TeX{} was one of the first softwares to
be released as a free software. HTML has not yet matured to represent
mathematical content. MathML is supported by Mozilla. Webskit browsers started
supporting  then dropped and I would refrain to comment on commercial browser's support which do not even comply to starndard because they think they are the standards tehmselves and others will bend to their will.

Now as this book is being written using \LaTeX{}~ so obviously Leslie Lamport
and all the people involved with it have my thanks along with Donald Knuth. I
use Emacs with Auctex and hope that someday I will use it in a much more
productive way someday.

I have used TikZ as a tool for drawing all the diagrams. It is a wonderful
package and works very nicely. I have great appreciation for its author Till
Tantau.

%For syntax highlighting I have used \texttt{Verbatim} \LaTeX{} package which
%uses \texttt{pygments} as its backend. You can modify it and make it look
%different both at \LaTeX{} and \texttt{pygments} level. Thanks to the
%respective package authors.

For errors and suggestion please email me at
\href{mailto:shivshankar.dayal@gmail.com}{shivshankar.dayal@gmail.com} where I
will try to respond to each mail as
much as possible. Please use your real names in email not something like
coolguy. As an alternative you can report it at
\url{https://10hash.com} where a question/answer website is maintained.
\begin{flushright}
Shiv Shankar Dayal\\
Nalanda,\\
India, 2015
\end{flushright}
