\chapter{Console I/O}
IO stands for input/output. C provides only mechanism to interact through
console using its standard library. C does not provide ways to have GUI
although that is possible with various GUI libraries most notable being
GTK. However, discussing about GTK is out of scope of this book. In this
chapter we will focus on console output facilities of C because any program we
write at this stage will be meaningless if it has no input/output. Typically
when we interact with a C program we give input using keyboard which is also
referred as \texttt{stdin} stream. The output is monitor or \texttt{stdout}
stream. There is one more stream \texttt{stderr} which is generally redirected
to monitor or a log file. For historical reasons these are known as
\texttt{FILE} stream which represents the datatype of these
streams. \texttt{FILE} is capable of representing other streams which are disk
based for example a file on your hard drive. There are more type of input
devices on a modern computer. For example, network i/o is there. Whenever you
browse web or download a file through your intenet connection network i/o comes
into play. There is an opengroup
which specifies functions for network related functions. Operating systems
like GNU/Linux are POSIX compatible which defines how network i/o will be
used. Even a printer is a special output device, a camera input, speakers
output, microphone input and so on. In this books we are concerned with
keyboard input, output on monitor and i/o using files. Other types of i/os are
out of scope of this book.

However, before we go on with i/o I would
like to present C's memory model which will be needed by our discussion of i/o
related functions. However, if things do not make sense even then please go
through it and come later to understand more. 