\chapter{Operators and Expressions}
Operators and expressions are in the core of every programming language. They
form the major part of BNF grammar. They also decide how the syntax will look
like. You as a programmer will spend considerable time using C operators. C has
sevral type of operators like arithmetic operators, relational operators,
bitwise operators, unary operators, logical operators to name some of them.
Since C was first of very poopular structured general-pupose lnguages therefore
many modern language use almost all the operators and supplement with their own.
It is needless to say that to become a good programmar you must know all the
operators of C and know where to use which one as it may decide performance,
readability, simplicity of your code. Whenever you see array and pointer in
following sections just plow through them. All will be clear soon.

Before we can proceed to discuss operators and expressions I will explain
scope, linkage, namespaces and storage durations which can be applied to
variables. These are given in specification starting in \S(6.2.1) and ending at
\S(6.2.4).


Whenever operators and expressions come in picture you may have a set of mixed
data then to perform opration data is converted from one type to another. This
is known as "Usual Arithmetic Conversion", which I am going to tell you next.