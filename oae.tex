\chapter{Operators and Expressions}
Operators and expressions are in the core of every programming language. They
form the major part of BNF grammar. They also decide how the syntax will look
like. You as a programmer will spend considerable time using C operators. C has
sevral type of operators like arithmetic operators, relational operators,
bitwise operators, unary operators, logical operators to name some of them.
Since C was first of very poopular structured general-pupose lnguages therefore
many modern language use almost all the operators and supplement with their own.
It is needless to say that to become a good programmar you must know all the
operators of C and know where to use which one as it may decide performance,
readability, simplicity of your code. Whenever you see array and pointer in
following sections just plow through them. All will be clear soon.

Before we can proceed to discuss operators and expressions I will explain
scope, linkage and storage durations which can be applied to
variables. These are given in specification starting in \S(iso.6.2.1) and ending at
\S(iso.6.2.4).

\section{Scope of an Identifier}
Till now we have seen plain variables and their identifiers. However, there are
other identifiers as well which will be discussed later. For now we will
consider scope of plain variables. In general there are three kinds of
scope. Global scope, function scope and block scope. Variables declared outside
any function have global scope and they persist throughout the lifetime of the
program. Variables declared inside functions at outermost level have function
scope and they live as long as function remains active. A block in C is marked
by braces(\{ and \}). Function bodies are also marked by this. Here I mean
blocks inside a function. Starting from C99 you can declare variables anywhere
inside a function and this block variables which have less lifetime than
functions are possible. We will see more of these when we see more code. Note
that identifiers can be reused in different scopes. For example, a loop index
integer identifer is repeated many times but every time it is a new
variable(We will see loops soon). Two idendifiers have same scope if and only
if their scope terminates at the same point.

\section{Linkages of an Identifier}
There are three different kinds of linkages. External, internal and
none. Global variables and functions have external linkage as long as they are
not static. If they are static then they have internal linkage. By external
linkage we mean that for a program which consists of multiple source code files
these functions and variable identifiers can be referred in files other than in
which they are declared. When functions and global variables are static
i.e. they have internal linkage they cannot be accessed in other source code
files.

The following identifiers have no linkage: an identifier declared to be
anything other than an variable or a function; an identifier declared to be a
function parameter; a block scope identifier for an object declared without the
storage-class specifier `extern`.

\section{Storage Duration of Objects}
There are four storage durations. Static, thread, automatic and
allocated. Here, we will not discuss thread which we will talk about later. A
static variable which is local to a function of global variable has static
duration and it lives in data segment in memory and has static storage
duration. A variable local to a function or block which is not dynamically
allocated on heap by using either of `malloc, calloc` or `realloc` has
automatic storage and has function or block has automatic storage and is
cleaned up automatically and it lives on stack. Allocated storage duration
varibales can persist as long as they want after allocation on heap by using
one of `malloc, calloc` and `realloc` as long as the name is kept
in scope and a corresponding `free` is not called on that name of the
variable. Now let us discuss operators and expressions.


Whenever operators and expressions come in picture you may have a set of mixed
data then to perform opration data is converted from one type to another. This
has an entire section devoted to it in specification at \S(iso.6.3). There are two
types of conversions. Many operators convert their operands silently which is
called ``implicit conversion'' and then we have cast operators
which we can use to explicitly convert values from one type to another which is
called ``explicit conversion''. We will first see implicit conversion.

\section{Arithmetic Conversions}
\subsection{Booleans, Integers and Characters}
These are all integral types. They have a rank or priority associated with them
which controls which value is converted to which one. Following comes from
\S(iso.6.3) for accuracy.

\begin{itemize}
\item[---] No two signed integer types shall have the same rank, even if they
  have the same representation.
\item[---] The rank of a signed integer type shall be greater than the rank of
  any signed integer type with less precision.
\item[---] The rank of \texttt{long long int} shall be greater than the rank of
  \texttt{long int,} which shall be greater than the rank of \texttt{int,}
  which shall be greater than the rank of \texttt{short int,} which shall be
  greater than the rank of \texttt{signed char}.
\item[---] The rank of any unsigned integer type shall equal the rank of the
  corresponding signed integer type, if any.
\item[---] The rank of any standard integer type shall be greater than the rank
  of any extended integer type with the same width.
\item[---] The rank of \texttt{char} shall equal the rank of \texttt{signed
    char} and \texttt{unsigned char}.\footnote{However, plain char is treated
    as signed char in gcc.}
\item[---] The rank of \texttt{\_Bool} shall be less than the rank of all other
  standard integer types. 
\item[---] The rank of any enumerated type shall equal the rank of the
  compatible integer type \S(iso.6.7.2.2).
\item[---]The rank of any extended signed integer type relative to another
  extended signed integer type with the same precision is
  implementation-defined, but still subject to the other rules for determining
  the integer conversion rank.
\item[---] For all integer types \texttt{T1, T2,} and \texttt{T3,} if
  \texttt{T1} has greater rank than \texttt{T2} and \texttt{T2} has greater
  rank than \texttt{T3,} then \texttt{T1} has greater rank than \texttt{T3}.
\end{itemize}

The following may be used in an expression wherever an int or unsigned int may
be used:

\begin{itemize}
\item[---] An object or expression with an integer type (other than
  \texttt{int} or \texttt{unsigned int}) whose integer conversion rank is less
  than or equal to the rank of \texttt{int} and \texttt{unsigned int}.
\item[---] A bit-field of type \texttt{\_Bool, int, signed int} or \texttt{unsigned int}.
\end{itemize}

If an \texttt{int} can represent all values of the original type (as restricted
by the width, for a bit-field(we will see these when we discuss structures and
unions)), the value is converted to an \texttt{int}; otherwise, it is converted
to an \texttt{unsigned int}. These are called the \textit{integer
  promotions}.\footnote{The integer promotions are applied only: as part of the
  usual arithmetic conversions, to certain argument expressions, to the
  operands of the unary \texttt{\textbf{+, -}} and \texttt{\textbf{\~}}
  operators, and to both operands of the shift operators, as specified by their
  respective subclauses.} All other types are unchanged by theinteger
promotions.

The integer promotions preserve value including sign. As discussed earlier, whether a
``plain'' \texttt{char} is treated as signed is implementation-defined.

\subsection{Boolean Types}
All values are convert to \texttt{\_Bool}. If non-zero then it is 1 else 0 or
\texttt{true} and \texttt{false} respectively.\footnote{NaNs are not equal to 0
  and thus convert to 1.}

\subsection{Signed and Unsigned Integers}
When a value with integer type is converted to another integer type other than
\texttt{\_Bool}, if the value can be represented by the new type, it is
unchanged.

Otherwise, if the new type is unsigned, the value is converted by repeatedly
adding or subtracting one more than the maximum value that can be represented
in the new type until the value is in the range of the new type.\footnote{The
  rules describe arithmetic on the mathematical value, not the value of a given
  type of expression.}

Otherwise, the new type is signed and the value cannot be represented in it;
either the result is implementation-defined or an implementation-defined signal
is raised.

\subsection{Real Floating and Integer}
When a finite value of real floating type is converted to an integer type other
than \texttt{\_Bool,} the fractional part is discarded (i.e., the value is
truncated toward zero). If the value of the integral part cannot be represented
by the integer type, the behavior is undefined.\footnote{The remaindering
  operation performed when a value of integer type is converted to unsigned
  type need not be performed when a value of real floating type is converted to
  unsigned type. Thus, the range of portable real floating values is (−1,
  \texttt{\textbf{U}}\textit{type}\_\texttt{MAX+1}).}

When a value of integer type is converted to a real floating type, if the value
being converted can be represented exactly in the new type, it is unchanged. If
the value being converted is in the range of values that can be represented but
cannot be represented exactly, the result is either the nearest higher or
nearest lower representable value, chosen in an implementation-defined
manner. If the value being converted is outside the range of values that can be
represented, the behavior is undefined. Results of some implicit conversions
may be represented in greater range and precision than that required by the new
type (see \S(iso.6.3.1.8) and \S(iso.6.8.6.4)).

\subsection{Complex Types}
When a value of complex type is converted to another complex type, both the
real and imaginary parts follow the conversion rules for the corresponding real
types.

\subsection{Real and Complex}
When a value of real type is converted to a complex type, the real part of the
complex result value is determined by the rules of conversion to the
corresponding real type and the imaginary part of the complex result value is a
positive zero or an unsigned zero.

When a value of complex type is converted to a real type, the imaginary part of
the complex value is discarded and the value of the real part is converted
according to the conversion rules for the corresponding real type.

\subsection{Usual Arithmetic Conversions}
Many operators that expect operands of arithmetic type(characters, integers and
floating-point numbers) cause conversions and
yield result types in a similar way. The purpose is to determine a
\textit{common real type} for the operands and result. For the specified
operands, each operand is converted, without change of type domain, to a type
whose corresponding real type is the common real type. Unless explicitly stated
otherwise, the common real type is also the corresponding real type of the
result, whose type domain is the type domain of the operands if they are the
same, and complex otherwise. This pattern is called the \textit{usual
  arithmetic conversions}:

\setlength{\leftskip}{1.5cm}

\noindent First, if the corresponding real type of either operand is \texttt{long
  double,} the other operand is converted, without change of type domain, to a
type whose corresponding real type is \texttt{long double}.


\noindent Otherwise, if the corresponding real type of either operand is \texttt{double,}
the other operand is converted, without change of type domain, to a type whose
corresponding real type is \texttt{double}.


\noindent Otherwise, if the corresponding real type of either operand is \texttt{float,}
the other operand is converted, without change of type domain, to a type whose
corresponding real type is \texttt{float}.


\noindent Otherwise, the integer promotions are performed on both operands. Then the
following rules are applied to the promoted operands:

\setlength{\leftskip}{3cm}

\noindent If both operands have the same type, then no further conversion is needed.


\noindent Otherwise, if both operands have signed integer types or both have unsigned
integer types, the operand with the type of lesser integer conversion rank is
converted to the type of the operand with greater rank.


\noindent Otherwise, if the operand that has unsigned integer type has rank greater or
equal to the rank of the type of the other operand, then the operand with
signed integer type is converted to the type of the operand with unsigned
integer type.


\noindent Otherwise, if the type of the operand with signed integer type can represent
all of the values of the type of the operand with unsigned integer type, then
the operand with unsigned integer type is converted to the type of the
operand with signed integer type.


\noindent Otherwise, both operands are converted to the unsigned integer type
corresponding to the type of the operand with signed integer type.


\setlength{\leftskip}{0cm}
The values of floating operands and of the results of floating expressions may
be represented in greater range and precision than that required by the type;
the types are not changed thereby.\footnote{The cast and assignment operators
  are still required to remove extra range and precision.}

There is also a concept called \textit{lvalue}\footnote{The name ``lvalue''
  comes originally from the assignment expression \texttt{E1 = E2,} in which
  the left operand \texttt{E1} is required to be a (modifiable) lvalue. It is
  perhaps better considered as representing an object ``locator value''. What
  is sometimes called ``rvalue'' is in this International Standard described 
  as the ``value of an expression''.}. An lvalue is a value whose
address can be taken. A modifiable lvalue is a value lvalue that does not have
array type, does not have an incomplete type i.e. void, does not have a
const-qualified type, and if it is a structure or union, does not have any
member (including, recursively, any member or element of all contained
aggregates or unions) with a const-qualified type.

\section{Additive Operators}
For addition \texttt{+} is used as symbol and for subtraction \texttt{-} is
used as symbol. For addition both operands can be pointers(do not worry about
these for now) or integers or characters or floating-point numbers. In other
case one can be a pointer to a complete object and another can be an interger.

For subtraction both operands should be pointers or left one can be pointer and
right one can be integer else both operands can be arithmetic type
i.e. characters, integers or floating-point numbers.

Usual arithmetic conversions are performed on operands which we have discussed
above. For now we will only consider arithemtic types and not pointers.

\section{Multiplicative operators}
There are three multiplicative operators. For multiplication \texttt{*} is
used. For division / is used and for calculating remainder or modulus
\texttt{\%} is used. For all these operators usual arithmetic conversions are
performed and operands must be of arithmetic type. \texttt{\%} can only be
applied to integral types i.e. characters and integers but not to
floating-point numbers as \texttt{/} operation's result contain the fraction
part which forms the remainder for \texttt{\%} in case of both the operands
being integeral type.

Let us see a program to understand these operators clearly.
\begin{Verbatim}[frame=single]
#include <stdio.h>

int main()
{
  int i = 10;
  float f= 6.45;
  char c = 'A';
  int iResult = 0;
  float fResult = 0.0;
  char cResult = '\0';

  cResult = c + i;
  printf("cResult = %c\n", cResult);
  cResult = cResult - 5;
  printf("cResult = %c\n", cResult);

  iResult = i - 10;
  printf("iResult = %d\n", iResult);
  iResult = i * c;
  printf("iResult = %d\n", iResult);
  iResult = (i + c)/3;
  printf("Result = %d\n", iResult);
  iResult = (i + c)%2;
  printf("iesult = %d\n", iResult);

  fResult = f * 2.12;
  printf("fesult = %f\n", fResult);
  fResult = f - i;
  printf("fesult = %f\n", fResult);  
  fResult = f / 1.12;
  printf("fesult = %f\n", fResult);
  fResult = 1 % 3;
  printf("fesult = %f\n", fResult);

  return 0;
}
\end{Verbatim}

and the output is:

\begin{Verbatim}
cResult = K
cResult = F
iResult = 0
iResult = 650
Result = 25
iesult = 1
fesult = 13.674000
fesult = -3.550000
fesult = 5.758928
fesult = 1.000000
\end{Verbatim}

First cResult is sumof \texttt{'A' + i} which is \texttt{'K'} as \texttt{'K'}
comes ten positions after \texttt{'A'} in ASCII table. Then we subtract five and
go back to \texttt{'F'}.

First \texttt{iReasult} is \texttt{10 - i} where value of \texttt{i} is 10
hence result is 0. Next we multiply \texttt{i} with \texttt{c} which
contains \texttt{'A'} who has got ASCII value of 65 and result becomes
650. Then We take sum of \texttt{'A'} and  \texttt{i} and divide by 3 so the
result is 25 as it is a division of 75 by 3. Next we use modulus operator and
remainder is 1. Note that in case of / and \% if denominator is zero the
behavior is undefined.

Consider a different program:

\begin{Verbatim}[frame=single]
#include <stdio.h>
#include <limits.h>

int main()
{
  int i=INT_MAX;
  int j = INT_MAX;

  printf("%d\n", i + j);
  printf("%ld", (long)i + (long)j);

  return 0;
}
\end{Verbatim}

and the output is:
\\\\\texttt{-2\\
4294967294\\\\}
Here, \texttt{INT\_MAX} is 2147483647 which is nothing but $2^{31} - 1$ for
32-bit intergers. Now if we add these it will cause integer overflow. Thus
specification gives the compiler freedom that you can wrap around the sum
around the maximum value. Thus if you go on counting in 2's complement and wrap
after \texttt{INT\_MAX} and count \texttt{INT\_MAX} times again then you will
get -2. Now the question is how can we add these numbers. We can promote them
to a bigger data types say \texttt{long} by using cast operators which we will
see later. Casting signed type to unsigned types is not advisable because sign
conversions will end up having values which are not intutive.

Similarly floating-point arithmetic can also be done.

\section{Relational Operators}
