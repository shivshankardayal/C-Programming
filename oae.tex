\chapter{Operators and Expressions}
Operators and expressions are in the core of every programming language. They
form the major part of BNF grammar. They also decide how the syntax will look
like. You as a programmer will spend considerable time using C operators. C has
several type of operators like arithmetic operators, relational operators,
bitwise operators, unary operators, logical operators to name some of them.
Since C was first of very popular structured general-purpose languages therefore
many modern language use almost all the operators and supplement with their own.
It is needless to say that to become a good programmer you must know all the
operators of C and know where to use which one as it may decide performance,
readability, simplicity of your code. Whenever you see array and pointer in
following sections just plow through them. All will be clear soon.

Before we can proceed to discuss operators and expressions I will explain
scope, linkage and storage durations which can be applied to
variables. These are given in specification starting in \S(iso.6.2.1) and ending at
\S(iso.6.2.4).

\section{Scope of an Identifier}
Till now we have seen plain variables and their identifiers. However, there are
other identifiers as well which will be discussed later. For now we will
consider scope of plain variables. In general there are three kinds of
scope. Global scope, function scope and block scope. Variables declared outside
any function have global scope and they persist throughout the lifetime of the
program. Variables declared inside functions at outermost level have function
scope and they live as long as function remains active. A block in C is marked
by braces(\{ and \}). Function bodies are also marked by this. Here I mean
blocks inside a function. Starting from C99 you can declare variables anywhere
inside a function and this block variables which have less lifetime than
functions are possible. We will see more of these when we see more code. Note
that identifiers can be reused in different scopes. For example, a loop index
integer identifier is repeated many times but every time it is a new
variable(We will see loops soon). Two identifiers have same scope if and only
if their scope terminates at the same point.

\section{Linkages of an Identifier}
There are three different kinds of linkages. External, internal and
none. Global variables and functions have external linkage as long as they are
not static. If they are static then they have internal linkage. By external
linkage we mean that for a program which consists of multiple source code files
these functions and variable identifiers can be referred in files other than in
which they are declared. When functions and global variables are static
i.e. they have internal linkage they cannot be accessed in other source code
files.

The following identifiers have no linkage: an identifier declared to be
anything other than an variable or a function; an identifier declared to be a
function parameter; a block scope identifier for an object declared without the
storage-class specifier `extern`.

\section{Storage Duration of Objects}
There are four storage durations. Static, thread, automatic and
allocated. Here, we will not discuss thread which we will talk about later. A
static variable which is local to a function of global variable has static
duration and it lives in data segment in memory and has static storage
duration. A variable local to a function or block which is not dynamically
allocated on heap by using either of `malloc, calloc` or `realloc` has
automatic storage and has function or block has automatic storage and is
cleaned up automatically and it lives on stack. Allocated storage duration
variables can persist as long as they want after allocation on heap by using
one of `malloc, calloc` and `realloc` as long as the name is kept
in scope and a corresponding `free` is not called on that name of the
variable. Now let us discuss operators and expressions.


Whenever operators and expressions come in picture you may have a set of mixed
data then to perform oration data is converted from one type to another. This
has an entire section devoted to it in specification at \S(iso.6.3). There are two
types of conversions. Many operators convert their operands silently which is
called ``implicit conversion'' and then we have cast operators
which we can use to explicitly convert values from one type to another which is
called ``explicit conversion''. We will first see implicit conversion.

\section{Arithmetic Conversions}

\subsection{Booleans, Integers and Characters}
These are all integral types. They have a rank or priority associated with them
which controls which value is converted to which one. Following comes from
\S(iso.6.3) for accuracy.

\begin{itemize}
\item[---] No two signed integer types shall have the same rank, even if they
  have the same representation.
\item[---] The rank of a signed integer type shall be greater than the rank of
  any signed integer type with less precision.
\item[---] The rank of \texttt{long long int} shall be greater than the rank of
  \texttt{long int,} which shall be greater than the rank of \texttt{int,}
  which shall be greater than the rank of \texttt{short int,} which shall be
  greater than the rank of \texttt{signed char}.
\item[---] The rank of any unsigned integer type shall equal the rank of the
  corresponding signed integer type, if any.
\item[---] The rank of any standard integer type shall be greater than the rank
  of any extended integer type with the same width.
\item[---] The rank of \texttt{char} shall equal the rank of \texttt{signed
    char} and \texttt{unsigned char}.\footnote{However, plain char is treated
    as signed char in gcc.}
\item[---] The rank of \texttt{\_Bool} shall be less than the rank of all other
  standard integer types. 
\item[---] The rank of any enumerated type shall equal the rank of the
  compatible integer type \S(iso.6.7.2.2).
\item[---]The rank of any extended signed integer type relative to another
  extended signed integer type with the same precision is
  implementation-defined, but still subject to the other rules for determining
  the integer conversion rank.
\item[---] For all integer types \texttt{T1, T2,} and \texttt{T3,} if
  \texttt{T1} has greater rank than \texttt{T2} and \texttt{T2} has greater
  rank than \texttt{T3,} then \texttt{T1} has greater rank than \texttt{T3}.
\end{itemize}

The following may be used in an expression wherever an int or unsigned int may
be used:

\begin{itemize}
\item[---] An object or expression with an integer type (other than
  \texttt{int} or \texttt{unsigned int}) whose integer conversion rank is less
  than or equal to the rank of \texttt{int} and \texttt{unsigned int}.
\item[---] A bit-field of type \texttt{\_Bool, int, signed int} or \texttt{unsigned int}.
\end{itemize}

If an \texttt{int} can represent all values of the original type (as restricted
by the width, for a bit-field(we will see these when we discuss structures and
unions)), the value is converted to an \texttt{int}; otherwise, it is converted
to an \texttt{unsigned int}. These are called the \textit{integer
  promotions}.\footnote{The integer promotions are applied only: as part of the
  usual arithmetic conversions, to certain argument expressions, to the
  operands of the unary \texttt{\textbf{+, -}} and \texttt{\textbf{\~}}
  operators, and to both operands of the shift operators, as specified by their
  respective subclauses.} All other types are unchanged by the integer
promotions.

The integer promotions preserve value including sign. As discussed earlier, whether a
``plain'' \texttt{char} is treated as signed is implementation-defined.

\subsection{Boolean Types}
All values are convert to \texttt{\_Bool}. If non-zero then it is 1 else 0 or
\texttt{true} and \texttt{false} respectively.\footnote{NaNs are not equal to 0
  and thus convert to 1.}

\subsection{Signed and Unsigned Integers}
When a value with integer type is converted to another integer type other than
\texttt{\_Bool}, if the value can be represented by the new type, it is
unchanged.

Otherwise, if the new type is unsigned, the value is converted by repeatedly
adding or subtracting one more than the maximum value that can be represented
in the new type until the value is in the range of the new type.\footnote{The
  rules describe arithmetic on the mathematical value, not the value of a given
  type of expression.}

Otherwise, the new type is signed and the value cannot be represented in it;
either the result is implementation-defined or an implementation-defined signal
is raised.

\subsection{Real Floating and Integer}
When a finite value of real floating type is converted to an integer type other
than \texttt{\_Bool,} the fractional part is discarded (i.e., the value is
truncated toward zero). If the value of the integral part cannot be represented
by the integer type, the behavior is undefined.\footnote{The remaindering
  operation performed when a value of integer type is converted to unsigned
  type need not be performed when a value of real floating type is converted to
  unsigned type. Thus, the range of portable real floating values is (−1,
  \texttt{\textbf{U}}\textit{type}\_\texttt{MAX+1}).}

When a value of integer type is converted to a real floating type, if the value
being converted can be represented exactly in the new type, it is unchanged. If
the value being converted is in the range of values that can be represented but
cannot be represented exactly, the result is either the nearest higher or
nearest lower representable value, chosen in an implementation-defined
manner. If the value being converted is outside the range of values that can be
represented, the behavior is undefined. Results of some implicit conversions
may be represented in greater range and precision than that required by the new
type (see \S(iso.6.3.1.8) and \S(iso.6.8.6.4)).

\subsection{Complex Types}
When a value of complex type is converted to another complex type, both the
real and imaginary parts follow the conversion rules for the corresponding real
types.

\subsection{Real and Complex}
When a value of real type is converted to a complex type, the real part of the
complex result value is determined by the rules of conversion to the
corresponding real type and the imaginary part of the complex result value is a
positive zero or an unsigned zero.

When a value of complex type is converted to a real type, the imaginary part of
the complex value is discarded and the value of the real part is converted
according to the conversion rules for the corresponding real type.

\section{Primary Expressions}
An identifier, a constant, a string literal, a parenthesized expression and a
generic selection are all examples of primary expression.

\subsection{Usual Arithmetic Conversions}
Many operators that expect operands of arithmetic type(characters, integers and
floating-point numbers) cause conversions and
yield result types in a similar way. The purpose is to determine a
\textit{common real type} for the operands and result. For the specified
operands, each operand is converted, without change of type domain, to a type
whose corresponding real type is the common real type. Unless explicitly stated
otherwise, the common real type is also the corresponding real type of the
result, whose type domain is the type domain of the operands if they are the
same, and complex otherwise. This pattern is called the \textit{usual
  arithmetic conversions}:

\setlength{\leftskip}{1.5cm}

\noindent First, if the corresponding real type of either operand is \texttt{long
  double,} the other operand is converted, without change of type domain, to a
type whose corresponding real type is \texttt{long double}.


\noindent Otherwise, if the corresponding real type of either operand is \texttt{double,}
the other operand is converted, without change of type domain, to a type whose
corresponding real type is \texttt{double}.


\noindent Otherwise, if the corresponding real type of either operand is \texttt{float,}
the other operand is converted, without change of type domain, to a type whose
corresponding real type is \texttt{float}.


\noindent Otherwise, the integer promotions are performed on both operands. Then the
following rules are applied to the promoted operands:

\setlength{\leftskip}{3cm}

\noindent If both operands have the same type, then no further conversion is needed.


\noindent Otherwise, if both operands have signed integer types or both have unsigned
integer types, the operand with the type of lesser integer conversion rank is
converted to the type of the operand with greater rank.


\noindent Otherwise, if the operand that has unsigned integer type has rank greater or
equal to the rank of the type of the other operand, then the operand with
signed integer type is converted to the type of the operand with unsigned
integer type.


\noindent Otherwise, if the type of the operand with signed integer type can represent
all of the values of the type of the operand with unsigned integer type, then
the operand with unsigned integer type is converted to the type of the
operand with signed integer type.


\noindent Otherwise, both operands are converted to the unsigned integer type
corresponding to the type of the operand with signed integer type.


\setlength{\leftskip}{0cm}
The values of floating operands and of the results of floating expressions may
be represented in greater range and precision than that required by the type;
the types are not changed thereby.\footnote{The cast and assignment operators
  are still required to remove extra range and precision.}

There is also a concept called \textit{lvalue}\footnote{The name ``lvalue''
  comes originally from the assignment expression \texttt{E1 = E2,} in which
  the left operand \texttt{E1} is required to be a (modifiable) lvalue. It is
  perhaps better considered as representing an object ``locator value''. What
  is sometimes called ``rvalue'' is in this International Standard described 
  as the ``value of an expression''.}. An lvalue is a value whose
address can be taken. A modifiable lvalue is a value lvalue that does not have
array type, does not have an incomplete type i.e. void, does not have a
const-qualified type, and if it is a structure or union, does not have any
member (including, recursively, any member or element of all contained
aggregates or unions) with a const-qualified type.

\section{Additive Operators}
For addition \texttt{+} is used as symbol and for subtraction \texttt{-} is
used as symbol. For addition both operands can be pointers(do not worry about
these for now. We will refer to this later) or integers or characters or 
floating-point numbers. In other
case one can be a pointer to a complete object and another can be an interger.

For subtraction both operands should be pointers or left one can be pointer and
right one can be integer else both operands can be arithmetic type
i.e. characters, integers or floating-point numbers.

Usual arithmetic conversions are performed on operands which we have discussed
above. For now we will only consider arithmetic types and not pointers.

\section{Multiplicative operators}
There are three multiplicative operators. For multiplication \texttt{*} is
used. For division / is used and for calculating remainder or modulus
\texttt{\%} is used. For all these operators usual arithmetic conversions are
performed and operands must be of arithmetic type. \texttt{\%} can only be
applied to integral types i.e. characters and integers but not to
floating-point numbers as \texttt{/} operation's result contain the fraction
part which forms the remainder for \texttt{\%} in case of both the operands
being integral type.

Let us see a program to understand these operators clearly.
\begin{minted}[frame=single]{c}
#include <stdio.h>

int main()
{
  int i = 10;
  float f= 6.45;
  char c = 'A';
  int iResult = 0;
  float fResult = 0.0;
  char cResult = '\0';

  cResult = c + i;
  printf("cResult = %c\n", cResult);
  cResult = cResult - 5;
  printf("cResult = %c\n", cResult);

  iResult = i - 10;
  printf("iResult = %d\n", iResult);
  iResult = i * c;
  printf("iResult = %d\n", iResult);
  iResult = (i + c)/3;
  printf("Result = %d\n", iResult);
  iResult = (i + c)%2;
  printf("iesult = %d\n", iResult);

  fResult = f * 2.12;
  printf("fesult = %f\n", fResult);
  fResult = f - i;
  printf("fesult = %f\n", fResult);  
  fResult = f / 1.12;
  printf("fesult = %f\n", fResult);
  fResult = 1 % 3;
  printf("fesult = %f\n", fResult);

  return 0;
}
\end{minted}

and the output is:

\begin{Verbatim}
cResult = K
cResult = F
iResult = 0
iResult = 650
Result = 25
iesult = 1
fesult = 13.674000
fesult = -3.550000
fesult = 5.758928
fesult = 1.000000
\end{Verbatim}

First cResult is sumof \texttt{'A' + i} which is \texttt{'K'} as \texttt{'K'}
comes ten positions after \texttt{'A'} in ASCII table. Then we subtract five and
go back to \texttt{'F'}.

First \texttt{iReasult} is \texttt{10 - i} where value of \texttt{i} is 10
hence result is 0. Next we multiply \texttt{i} with \texttt{c} which
contains \texttt{'A'} who has got ASCII value of 65 and result becomes
650. Then We take sum of \texttt{'A'} and  \texttt{i} and divide by 3 so the
result is 25 as it is a division of 75 by 3. Next we use modulus operator and
remainder is 1. Note that in case of / and \% if denominator is zero the
behavior is undefined.

Consider a different program:

\begin{minted}[frame=single]{c}
#include <stdio.h>
#include <limits.h>

int main()
{
  int i=INT_MAX;
  int j = INT_MAX;

  printf("%d\n", i + j);
  printf("%ld", (long)i + (long)j);

  return 0;
}
\end{minted}

and the output is:
\\\\\texttt{-2\\
4294967294\\\\}
Here, \texttt{INT\_MAX}(found in header \texttt{limits.h}) is 2147483647 which 
is nothing but $2^{31} - 1$ for
32-bit integers. Now if we add these it will cause integer overflow. Thus
specification gives the compiler freedom that you can wrap around the sum
around the maximum value. Thus if you go on counting in 2's complement and wrap
after \texttt{INT\_MAX} and count \texttt{INT\_MAX} times again then you will
get -2. Now the question is how can we add these numbers. We can promote them
to a bigger data types say \texttt{long} by using cast operators which we will
see later. For the curious one if you are wondering how can we represent 
integers larger than \texttt{long long} then the answer lies in many methods. 
You can implement a linked list or use  a multiple-precision library like 
\href{https://gmplib.org/}{GMP}. Casting signed type to unsigned types is not 
advisable because sign conversions will end up having values which are not 
intuitive.

You should be careful when using \texttt{*} or \texttt{\/} with \texttt{+} or
\texttt{-} because \texttt{*} and \texttt{/} have higher 
priority than \texttt{+} or \texttt{-}. For example, \texttt{2+3*4} will
evaluate for 14 while you may 
have intended it for 20. In such cases it is mandatory to override the
precedence using parentheses like \texttt{(2+3)*4}. \texttt{\/} with two
integral operands cause 
integer devision and remainder is lost. To obtain remainder you can use
\texttt{\%} operator which is also known as modulus or mod operator.

Similarly, floating-point arithmetic can also be done.
Note that you cannot use modulus operator if either of the operands are
floating-point numbers as it will make no sense because of data type promotion
rules. Here data type promotion rule says smaller data types will be converted
to bigger data types. Also, if there is a data type on left side of assignment
the result of applying the operator to operands will be converted to the type
of that. chars are promoted to ints, ints are promoted to floats anf floats to
double. The point is that conversion will try to keep as much data as
possible. 

You should be also careful while choosing your data types. For example, if you
take two short integers and add them then the sum may overflow the range of
short integers. For example, sum of \texttt{short int si1= 100; short int
  si2=100;} will fit nicely in short integer range because sum is 200 but then
if it goes beyond 32,767 then it will overflow and will become either a zero or
a negative number. In such cases you can cast short integers to a larger data
type like \texttt{int} and put sum their. You can see it as a disadvantage as
you have to be very careful about what your data type has to be. In previous
example if you know that the two short integer's sum is going to fit in a short
integer then you can use a short integer for sum for sure. But since you may
not know who will be using your program most of the time and user input can
never be trusted it is always better to program defensively.

\section{Relational Operators}
There are four relational operators which are used to compare the value. There 
are two additional equality operators, which we will see next, are used for 
checking equality between its operands. The four relational operators are 
\texttt{<, <=, >, >=}. \texttt{<} and \texttt{>} are simple. They mean less 
than and greater than. \texttt{<=} checks if left operand is less than or equal 
to the right operand. Similarly, \texttt{>=} is used for checking if left 
operand is greater than or equal to right operand. The constraint or condition 
on operands is both operands will be real types i.e. characters. integers or 
floating-point numbers or both operands will be pointers to qualified or 
unqualified versions of compatible types(again we will not worry about pointers 
here). The result of comparison is an \texttt{int} having a value 0 or 1. 
Consider the following program:

\begin{minted}[frame=single]{c}
#include <stdio.h>
#include <stdbool.h>

int main()
{
  int i = 4, j = 5;
  _Bool result = 0;

  result = i < j;
  printf("%d\n", result);

  result = i > j;
  printf("%d\n", result);

  result = i <= j;
  printf("%d\n", result);

  result = i >= j;
  printf("%d\n", result);

  return 0;
}
\end{minted}

and the output is::
\\\\\texttt{1\\
0\\
1\\
0\\\\}
We could have used \texttt{char} data type as well since they are fundamentally 
integral types.


Note that you should not apply these to floating-point data types as they may
not be represented correctly and two different entities have the same internal
representation. Consider following \texttt{double} values:
\\\\\texttt{3ff0 0000 0000 0000\textsubscript{16}   = 1\\
3ff0 0000 0000 0001\textsubscript{16}   $\approx$ 1.0000000000000002, 
the smallest 
number > 1\\
3ff0 0000 0000 0002\textsubscript{16} $\approx$ 1.0000000000000004\\\\}
As you can see 1.0000000000000003 cannot be represented correctly comparing it 
against given values which are at either end may throw up surprises. For 
example, consider the following program:

\begin{minted}[frame=single]{c}
#include <stdio.h>
#include <stdlib.h>

int main()
{
  if(1.0000000000000003 == 1.0000000000000002)
    printf("Surprise!\n");

  return 0;
}
\end{minted}

In this program equality operator is used, which is discussed in next section, 
along with \texttt{if} statement which is part of next chapter and ideally the  
string \texttt{Surprise!} should not print but it prints. Therefore, it is not 
suggested to compare two floating point values if you suspect them to be very 
close to each other.

For example, consider another program:
\begin{minted}[frame=single]{c}
#include <stdio.h>

int main()
{
  float f1=1.00000199999;
  float f2=1.000002;

  printf("%.20f %.20f\n", f1, f2);
  printf("%d\n", f1<f2);
  printf("%d\n", f1>f2);
  printf("%d\n", f1==f2);

  return 0;
}
\end{minted}
and the output is:
\\\\\texttt{1.00000202655792236328 1.00000202655792236328\\
0\\
0\\
1\\\\}
As you can clearly see the comparison output is definitely wrong and with good
reason. Since floating point numbers are valid till sixth digit of precision
the two floating point numbers compare equal when tested using the equality
operator \texttt{==} which you will see soon. However, two floating point
numbers can be compared with great degree of accuracy and more so with
\texttt{double} data type. Consider the program with same data as above but
with \texttt{double} as data type.

\begin{minted}[frame=single]{c}
#include <stdio.h>

int main()
{
  double d1=1.00000199999;
  double d2=1.000002;

  printf("%.20f %.20f\n", d1, d2);
  printf("%d\n", d1<d2);
  printf("%d\n", d1>d2);
  printf("%d\n", d1==d2);

  return 0;
}
\end{minted}
and the output is:
\\\\\texttt{1.00000199999000005668 1.00000200000000005751\\
1\\
0\\
0\\\\}
As you can see our result is correct. However, if the two doubles would have
been very-very close it would suffer the same fate as previous program and give
wrong output.

\section{Equality Operators}
There are two equlity operators == and !=. Equality operators are very much
similar to relational operation we have just  
discussed but their precedence is lower.\footnote{Because of the precedences, 
\texttt{a<b == c<d} is 1 whenever \texttt{a<b} and \texttt{c<d} have the same 
truth-value.} There are four constraints on operands 
of equality operators:
\begin{itemize}
  \item[---] both the operands are of arithmetic type.
	\item[---] both operands are pointers to qualified or unqualified versions 
	of compatible types;
	\item[---] one operand is a pointer to an object type and the other is a 
	pointer to a qualified or unqualified version of \texttt{void}; or
	\item[---] one operand is a pointer and the other is a null pointer 
	constant.
\end{itemize}
Once again we will not worry about pointers i.e. we can ignore last three 
constraints. An example program is given below:
\begin{minted}[frame=single]{c}
// Description : Demo of equality operator

#include <stdio.h>
#include <stdbool.h>
int main()
{
  int i = 4, j = 5;
  _Bool result = 0;

  result = i == j;
  printf("%d\n", result);

  result = i != j;
  printf("%d\n", result);

  return 0;
}
\end{minted}
and the output is:
\\\\\texttt{0\\
1}\\\\
The equality operator \texttt{==} test for equality for its two operands and
\texttt{!=} tests for the operands not being equal. The result of comparison is
boolean value. For \texttt{==} if the operands are equal then result is
\texttt{true} else \texttt{false}. For \texttt{!= the} result is \texttt{true}
if the operands do not compare equal and \texttt{false} if they compare equal.

\section{Increment and Decrement Operators}
There is one increment and one decrement operator. \texttt{++} and
\texttt{{-}{-}}. Both come in two forms prefix and postfix. First we will see
prefix versions then postfix ones. There is only one constraint on prefix
operators of these and that is the operand of the prefix increment or decrement
operator will have atomic, qualified or unqualified real or pointer type and
will be a modifiable lvalue. The prefix operator is usually more efficient as
compared to postfix operators but that may not be true always. These are
described in \S(iso.6.5.2.4) and \S(iso.6.5.3.1).
\begin{minted}[frame=single]{c}
// Description : Demo of increment decrement operators

#include <stdio.h>

int main()
{
  float f = 7.123;

  printf("%f\n", ++f);
  printf("%f\n", --f);
  printf("%f\n", f++);
  printf("%f\n", f--);
  printf("%f\n", f);

  return 0;
}
\end{minted}
and the output is:
\\\\\texttt{8.123000\\
7.123000\\
7.123000\\
8.123000\\
7.123000\\\\}
As you can see for postfix operations the result does not change immediately
like its prefix counterpart. The reason lies in the fact that the value
computation of the result is sequenced before the side effect of updating the
stored value of the operand. What this means is the computation may not happen
unless the operand's value is not being updated and is deferred.

\section{Logical Operators}
There are two such operators. \texttt{\&\&} logical AND and
and \texttt{||} locical OR. Both the operators have the same constraints and it
is that both the operands will have scalar type i.e. integral type,
floating-point types and pointers.

The \texttt{\&\&} operator gives 1 if both the operands are non-zero else
0. The result type is int. It is different from bitwise \texttt{\&} operator in
the sense that it guarantess left-to-right evaluation; if the second operand is
evaluated, there is a sequence point between the evaluations of the first and
second operands. If the first operand is 0 then the second operand is not
evaluated. This is known as ``short-circuit evaluation''.

The \texttt{||} operator gives 1 if any of operands are non-zero else it gives
0. Same as logical AND operator and unlike bitwise \texttt{|} operator it
guarantees left-to-right evaluation and same goes for sequence points. If first
operand is non-zero, the second is not evaluated.

\begin{minted}[frame=single]{c}
// Description : Demo of logical AND & OR operators

#include <stdio.h>
#include <stdbool.h>

int main()
{
  int i = 4, j = 5, k = 0;
  bool result;

  result = i&&j;
  printf("%d\n", result);

  result = i||j;
  printf("%d\n", result);

  result = k&&j;
  printf("%d\n", result);

  result = k||j;
  printf("%d\n", result);

  return 0;
}
\end{minted}
and the output is:
\\\\\texttt{1\\
1\\
0\\
1\\\\}
note the use of \texttt{bool} here instead of \texttt{\_Bool} which is a macro
defined in \texttt{stdbool.h}.

\section{Bitwise Operators}
I will describe bitwise operators in little detiail here as we will study these
in great detail in their own chapter of bit manipulation

There are three bitwise operators. \texttt{\&, |} and
\texttt{\textasciicircum{}}. AND, OR and EX-OR respectively. OR is also called
inclusive OR. These have the same constraints and it is that operands should be
integer types. The usual arithmetic conversions are performed on the
operands. It is not hard to understand why it cannot be applied to
floating-point if you remember the floating point number representation. 

\begin{minted}[frame=single]{c}
// Description : Demo of bitwise operators

#include &stdio.h>
#include &stdbool.h>

int main()
{
  int i = 4, j = 5;
  int result;

  result = i&j;
  printf("%d\n", result);

  result = i|j;
  printf("%d\n", result);

  result = i^j;
  printf("%d\n", result);

  return 0;
}
\end{minted}

and the output is:
\\\\\texttt{4\\
5\\
1\\\\}
The bit representation of 4 is 100 and that of 5 is 101. Thus, AND is 100, OR
is 101 and EX-OR is 001. The values are operated on a bit-by-bit basis.

\section{Bitwise Shift Operators}
These are \texttt{\textless\textless} and \texttt{\textgreater\textgreater}
known repsectively as left shift and right shift operators.
The constraint is same as other bitwise operators that operands should be
integers. The integer promotions are performed on each of the operands. Left
shift by \texttt{n} essentialy means multiplication by $2^n$ and right shift
means ddivision by $2^n$. The division is integer division.

\begin{minted}[frame=single]{c}
// Description : Demo of shift operators

#include <stdio.h>

int main()
{
  int i  = 4;
  char c ='A';
  int result;

  result = c<<i;
  printf("%d\n", result);

  result = c>>i;
  printf("%d\n", result);

  return 0;
}
\end{minted}

and the output is:
\\\\\texttt{1040\\
4\\\\}
ASCII value of 'A' is 65 thus $65*2^4$ is 1040 and $65/2^4$ is 4.

\section{Assignment Operators}
These are \texttt{=, *=, \/=, \%=, +=, -=, \textless\textless =,
  \textgreater\textgreater =, \&=,
  \textasciicircum=} and \texttt{|=}. The constraint is that left operand
should be modifiable lvalue. An assignment operator stores a value in the
object designated by the left operand. An assignment expression has the value
of the left operand after the assignment, but is not an lvalue. The type of an
assignment expression is the type of the left operand unless the left operand
has qualified type, in which case it is the unqualified version of the type of
the left operand. The side effect of updating the stored value of the left
operand is sequenced after the value computations of the left and right
operands. The evaluations of the operands are unsequenced.

\begin{minted}[frame=single]{c}
// Description: Demo of compound assignments.

#include <stdio.h>

int main()
{
  int i   = 3;
  int j   = 3;
  float f = 4.7;
  float result=0.0;

  result += i+f;
  printf("%f\n", result);

  result -= f;
  printf("%f\n", result);

  j <<= i;
  printf("%d\n", j);

  return 0;
}
\end{minted}

and the output is:
\\\\\texttt{7.700000\\
3.000000\\
24\\\\}
The compund assignment operators are nothing but a shorthand. Let us say
\texttt{a, b} are operands and \texttt{o} is an operator then \texttt{a o= b;}
is a shorthand for \texttt{a = a o b;}.

\section{Conditional Operators}
A conditional operator has three operands. It consists of three
expressions i.e. \texttt{expression1? expression2: expression3;} The
constraints are:

The first operand shall have scalar type.

One of the following shall hold for the second and third operands:
\begin{itemize}
\item[---] both operands have arithmetic type;
\item[---] both operands have the same structure or union type;
\item[---] both operands have void type;
\item[---] both operands are pointers to qualified or unqualified versions of
  compatible types;
\item[---] one operand is a pointer and the other is a null pointer constant;
  or
\item[---] one operand is a pointer to an object type and the other is a
  pointer to a qualified or unqualified version of \texttt{void}.
\end{itemize}

If the first expression is \texttt{true} then result is output of second
expression else it is third expression. Note that conditional operator does not
yield an lvalue. An example program is given below:

\begin{minted}[frame=single]{c}
// Description : Demo of conditional operator

#include <stdio.h>

int main()
{
  int i = (4 < 5)? 7:10;

  printf("%d\n", i);

  return 0;
}
\end{minted}
and the output is 7 as first expression is true.

\section{Comma Operator}
It is a very simple operator. The left operand of a comma operator is evaluated
as a void expression; there is a sequence point between its evaluation and that
of the right operand. Then the right operand is evaluated; the result has its
type and value. A comma operator does not give an lvalue. Consider the
following program:

\begin{minted}[frame=single]{c}
#include <stdio.h>

int main()
{
  int i;

  i = 1, 2;
  printf("%d\n", i);

  i = (1, 2);
  printf("%d\n", i);
  return 0;
}
\end{minted}

and the output is:
\\\\\texttt{1\\
2\\\\}
as you can see since command operator has least priority(refer
\autoref{table:paat} \nameref{table:paat}) for first assignment the output
has value 1 but then we can
override the priority and the value of comma operator i.e. right-most operand
gets assigned to \texttt{i}.

A comma operator does not yield lvalue. There are places where comma operator
cannot be used where comma is used to separate the items in context for example
arguments of a function or list initializers. However, a parethesized
expression can still be there which allows comma operators to be used in these
places.

\section{sizeof Operator}
You have already see sizeof operator in second chapter when we saw sizes of
data types. However here is the constraint: the sizeof operator will not be
applied to an expression that has function type or an incomplete type, to the
parenthesized name of such a type, or to an expression that designates a
bit-field member.

The sizeof operator yields the size (in bytes) of its operand, which may be an
expression or the parenthesized name of a type. The size is determined from the
type of the operand. The result is an integer. If the type of the operand is a
variable length array type, the operand is evaluated; otherwise, the operand is
not evaluated and the result is an integer constant.

When applied to an operand that has type \texttt{char, unsigned char} or
\texttt{signed char}, (or a qualified version thereof) the result is 1. When
applied to an operand that has array type, the result is the total number of
bytes in the array. When applied to an operand that has structure or union
type, the result is the total number of bytes in such an object, including
internal and trailing padding.

\section{Unary Operators}
 We have already seen unary prefix and postfix versions of increment and
 decrement operators earlier. There are some more unary operators like
 \texttt{+, -, \~{}} and \texttt{!}.

\texttt{+} and \texttt{-} are applied to arithmetic types. \texttt{+} does not
change the value of operand but promotions are applied to the type if
required. For example, an integer can be promoted to long integer. \texttt{-}
on the other hand negates the value of operand and again promotions are
applicable.

Operator \texttt{\~{}} results in bitwise complement of the operand. For
example, consider following program:

\begin{minted}[frame=single]{c}
#include <stdio.h>

int main()
{
  int i = 0;

  printf("%d\n", ~i);
  printf("%d\n", ~1);

  return 0;
}
\end{minted}

and the output is:
\\\\\texttt{-1\\
-2\\\\}
Operator \texttt{!} tests whether a value is \texttt{false} or
\texttt{true}. If it is \texttt{true} and \texttt{!} is applied then result is
\texttt{false} and reverse is also applicable. Since all values in C other than
0 and \texttt{NULL} pointer is considered true thus applying \texttt{!} for
other than these two results in false boolean value.

Now I am going to tell you
about operator precedence and associativity and then about grouping
parenthes. Given below is the table for operator precedence and associativity,
however, you may not be familiar with few of them but later you will be:

\begin{center}
  \begin{longtable}{|l|l|}
    \hline
    \textbf{Operators}&\textbf{Associativity}\\\hline
    () [] . -\textgreater~++ {-}{-} (postfix) & left-to-right\\\hline
    ++ {-}{-} + - (unary) ! \~{} (types) * \& sizeof & right-to-left\\\hline
    * \/ \% & left-to-right\\\hline
    + - (Addition\/Subtraction) & left-to-right\\\hline
    \textless\textless~\textgreater\textgreater & left-to-right\\\hline
    \textless~\textgreater~\textless=~\textgreater= & left-to-right\\\hline
    == != & left-to-right\\\hline
    \& & left-to-right\\\hline
    \textasciicircum{} & left-to-right\\\hline
    | & left-to-right\\\hline
    \&\& & left-to-right\\\hline
    || & left-to-right\\\hline
    assignment operators & left-to-right\\\hline
    , & left-to-right\\\hline
    \caption{Priority and associativity table}
    \label{table:paat}
  \end{longtable}
\end{center}

\section{Grouping Parentheses}
Grouping parentheses are used to override operator precedence and group
expressions. NEVER EVER try to memorize and rely on precedence of
operators. Always use grouping parentheses. Till now I have shown very simple
examples of operators; here are some complex ones:

\begin{minted}[frame=single]{c}
// Description: Demo of grouping parentheses

#include <stdio.h>

int main()
{
  printf("%f\n", 5.2*(3.7+2.3));
  printf("%d\n", ((4<5)||(7^5)));

  return 0;
}
\end{minted}

This small program shows you what can go wrong if you rely on memory. It allows
you do addition first and then multiplcation. Inner parentheses are evaluated
first then outer ones. This concludes our chapter on operators and
expressions.

Another simple example which I am giving again for comma operator to
explain comma operator and parentheses in conjunction is given below:

\begin{minted}[frame=single]{c}
#include <stdio.h>

int main()
{
  int i;

  i = 1, 2;
  printf("%d\n", i);

  i = (1, 2);
  printf("%d\n", i);

  return 0;
}
\end{minted}

Now as we know from precedence table given above that comma operator has least
priority thus for \texttt{i = 1, 2;} even though it evaluates to 2 the
assignment will happen first and 1 is stored in \texttt{i}. However, when we
use parentheses to override the precedence the comma operator gets higher
precedence and we get 2 as value of \texttt{i}.

\section{Cast Operators}
You can specifically convert a type to another type. For example, an integer
can be converted to floating-point explicitly. The constraints are follwoing:\\
Unless the type name specifies a void type, the type name shall specify atomic,
qualified, or unqualified scalar type, and the operand shall have scalar
type.\\
Conversions that involve pointers, other than where permitted by the
constraints of \S(iso.6.5.16.1), shall be specified by means of an explicit
cast.\\
A pointer type shall not be converted to any floating type. A floating type
shall not be converted to any pointer type.

If an expression is preceded by parenthesized type then the value of that
expression is converted to that type and it is a cast construct. A cast does
not yield lvalue. Cast can promote and demote the range as needed by the
conversion. An example is given below:

\begin{minted}[frame=single]{c}
// Description : Demo of cast operators

#include <stdio.h>
#include <limits.h>

int main()
{
  int i=INT_MAX, j=INT_MAX;

  printf("%d\n", i+j);
  printf("%ld\n", (long)(i+j));
  printf("%ld\n", (long)i + (long)j);
  return 0;
}
\end{minted}

and the output is:
\\\\\texttt{-2\\
-2\\
4294967294\\\\}
As you can see if sum does not fit the type it will rotate and will give
unexpected value. Also, casting the sum does not work because prentheses will
override the priority. But casting individual variables does work.
