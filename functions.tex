\chapter{Functions}
I know that you will readily agree with me if I say that humans get bored if
they have to do same things again and again. I know you get bored too and I too
get bored. We all. We as humans have this built-in nature that repititive
things are just not fit for us. Also, as a human being our capacity to
understand large things at once is difficult. We understand small-small things
and build large chunk based on those small things. Dennis Ritchie perhaps had
known this. I am saying because C has got something called functions. C
functions allow you to split a big logic into small ones and therefore
facilitating modular programming. They also form the basis of strutctured
programming the very base which made C popular. There is also something called
recursion which is a very poewrful tool. In this chapter we will also see how
to do multifile programming. I cannot emphasize much that how important it is
that you master the technique of functions well and not to mention function
pointers which can do the magic. I will show you the very glimpse only. I can
show you the way but walking on that is your job. It is upto you to do the
actual work. I have kept things simple and minimal with a pupose. I do not want
you to get bogged down with a thick and heavy book. All my examples are toy
examples but you have seen things can get somehwat complex.

We have already seen the special \texttt{main()} function.

Given below are typical code for a function:

\begin{Verbatim}[frame=single]
//function prototype
return-type function-name(argument list); //here varible names may be ommitted

//function body
return-type function-name(argument list) //variable names cannot be ommitted``
{
  //your code here

  //call some other function
  function-name(arugment-list-without-type);

  return value-of-return-type;
}
\end{Verbatim}
This might be a bit abstract but please bear it a bit.

\section{Pass by Value}

Consider a program which adds two numbers and let us say that you may need to
add lots of them.

\begin{minted}[frame=single]{c}
#include <stdio.h>

void add(int first_integer, int second_integer)
{
  printf("%d+%d=%d\n", first_integer, second_integer, first_integer + second_integer);
}

int main()
{
  int a=5, b=7;

  add(a, b);

  return 0;
}
\end{minted}
Note that you need function body before its use else you need at least a
function prototype before use. If you do not do so you will get a compiler
warnign. An example is given below:

\begin{minted}[frame=single]{c}
#include <stdio.h>

//not how argument names are not required
void add(int, int);

int main()
{
  int a=5, b=7;

  add(a, b);

  return 0;
}

void add(int first_integer, int second_integer)
{
  printf("%d+%d=%d\n", first_integer, second_integer, first_integer + second_integer);
}
\end{minted}
output is same as above.

When a function is called a stack frame is created for that function. The
control is passed to that function. When a return statement is encountered or
end of that function is reached, the function returns. The local variables of
the caller function are saved on stack and thus their value is restored from
that value unless the address of those variables are passed.

What you have seen just above is known as pass-by-value. In this case a copy of
parameters is made and passed on to called function by caller function. So, if
called function makes a change to values then those are not reflected back in
the caller function. As an example I will use famous example of swapping values
of two variables. First, I will show how pass-by-value works. So here is the
code:

\begin{minted}[frame=single]{c}
#include <stdio.h>

void swap(int, int);

int main()
{
  int a=5, b=7;

  printf("Before swap a=%d and b=%d\n", a, b);
  swap(a, b);
  printf("After swap a=%d and b=%d\n", a, b);

  return 0;
}

void swap(int first_integer, int second_integer)
{
  int temp=first_integer;
  first_integer=second_integer;
  second_integer=temp;
}
\end{minted}
and the output is:
\\\\\texttt{Before swap a=5 and b=7\\
After swap a=5 and b=7}

\section{Pass by Address}
\label{section:Pass by address}
The output of previous program was not exactly what we wanted. The solution is
to pass-by-address. When you the address to a called function, it receives
address in a pointer variable. Then if it modifies the value stored at that
address then it is reflected back in the caller. Let us see an example to
understand:

\begin{minted}[frame=single]{c}
#include <stdio.h>

void swap(int*, int*);

int main()
{
  int a=5, b=7;

  printf("Before swap a=%d and b=%d\n", a, b);
  swap(&a, &b);
  printf("After swap a=%d and b=%d\n", a, b);

  return 0;
}

void swap(int* first_int, int* second_int)
{
  int temp=*first_int;
  *first_int=*second_int;
  *second_int=temp;
}
\end{minted}
and the output is:
\\\\\texttt{Before swap a=5 and b=7\\
After swap a=7 and b=5\\\\}
When you pass address of a variable the change is maintained at that address
and thus when the control is returned the values are changed.

\section{Passing Arrays}
Arrays are always passed by address because if they are passed by value then
they will eat upprevious real-estate of little stack memory. Now since they
are passed by address by default base address of the array is passed. This
also implied that changes made to the array by the called function will be
reflected in caller after the call is complete. Passing arrays around various
functions is easy but requires extra piece of information to be sent to the
called function from column because the when an array is passed it is always
passed as a pointer or in other words the array becomes a pointer. Thus you
need to pass the length of array if you want the
called function to iterate or traverse over the array. Given below is a simple
example demonstrating this:

\begin{minted}[frame=single]{c}
#include <stdio.h>

void f(int *a, int n)
{
  for(size_t i=0; i<n; ++i)
    printf("%d\n", a[i]);
}

int main()
{
  int a[10]={0, 1, 2, 3, 4, 5, 6, 7, 8, 9};
 
  f(a, 10);
  
  return 0;
}
\end{minted}

and the output is easy to guess. The program prints 0 through 9, each on a
separate line.

However, this is not the only way to send an array to function. The other
solution could be to send start and end element as pointer. Since we are
already sending first element as pointer if we have end element's address we
can iterate over the array with ease. This is demonstrated in the next
program:

\begin{minted}[frame=single]{c}
#include <stdio.h>

void f(int *a, int *end )
{
  for(int *p=a; p!=end; ++p)
    printf("%d\n", *p);
  printf("%d\n", *end);
}

int main()
{
  int a[10]={0, 1, 2, 3, 4, 5, 6, 7, 8, 9};
 
  f(a, &a[9]);
  
  return 0;
}
\end{minted}
Note that we need to print last element separately because loop will not
execute its code for the last element. This is cumbersome but does equivalent
work as the previous program.

However, passing two dimensional array is different. If you want to pass an
array of the form \texttt{a[m][n]} then it would be passed as \texttt{int
  (*)[n]}. Thus passing is not that you will pass \texttt{int **}. Following
code shows how to pass and print a two dimensional array:

\begin{minted}[frame=single]{c}
void f(int a[][2], unsigned int fd, unsigned int sd)
{
  for(unsigned int i=0; i<fd; ++i)
  {
    for(unsigned int j=0; j<sd; ++j)
    {
      printf("%d ", a[i][j]);
    }
    printf("\n");
  }
}

int main()
{
  int a[5][2]={{0, 1}, {2, 3}, {4, 5}, {6, 7}, {8, 9}};
 
  f(a, 5, 2);
  
  return 0;
}
\end{minted}
and the output is:
\\\\\texttt{0 1\\
2 3\\
4 5\\
6 7\\
8 9\\
\\\\}
Note that array type changes to \texttt{int (*)[2]}, similarly an array of form
\texttt{a[l][m][n]} will change its type to \texttt{int (*)[m][n]}.

We have already seen pass by address but what if we want to pass to pointers
and change the pointer itself. In the example shown in the
\autoref{section:Pass by address} \nameref{section:Pass by address} we have two
variables on stack whose addresses were passed and the values were swapped
which was contained on those addresses. Can we extrapolate this to change
addresses themselves. Let us see an example:

\begin{minted}[frame=single]{c}
#include <stdio.h>

void f(int *p, int *q)
{
  int *temp = p;

  p = q;
  q = temp;
}

int main()
{
  int i=5, j=10;
  int *p=&i, *q=&j;

  f(p, q);

  printf("%d %d\n", *p, *q);
  return 0;
}
\end{minted}
and the output is:
\\\\\texttt{5 10\\\\}
Now as we say the pointers are still pointing to original values. The reason is
because when we call function \texttt{f} and pass addresses of \texttt{i} and
\texttt{j} then a copy of those addresses are sent. If we want to swap the
values by swapping pointers then we have to send addresses of pointers. The
program is shown below:

\begin{minted}[frame=single]{c}
#include <stdio.h>

void f(int **p, int **q)
{
  int *temp = *p;

  *p = *q;
  *q = temp;
}

int main()
{
  int i=5, j=10;
  int *p=&i, *q=&j;

  f(&p, &q);

  printf("%d %d\n", *p, *q);
  return 0;
}
\end{minted}

Now since the pointers themselves are swapped it is easy to guess the output
and it is \texttt{10 5}. Notice since we are sending addresses of pointers we
have to receive them using pointer to pointer syntax.

\section{Recursion}
In C recusion is the concept of a function calling itself. When a repeated
operation has to be preformed over a variable, recursion can be used. Recursion
simplifies the code a lot. Typically there is always a more effective iterative
solutions are available but there are certain cases where recursion is always
better than iteration. For example, traversal of trees where iteration is not
so effective as compared to recursion. For beginners it is hard to understand
recursion but once you understand it then it is not that hard to
understand. The first example I am going to give is that of factorials. The
formula for factorial is given by $n!=\prod_{k=1}^{n}i$  and recursive
definition of factorial is given by:

$$\begin{array}{rl}n!=\left\{\begin{array}{ll}1&\hspace{1em}\text{if
    n=0}\\[4pt](n-1)!\ast n&\hspace{1em}\text{if
    n>0}\end{array}\right.\end{array}$$

Note that every recursion has to be written carefully in thse sense that it
must have a termination condition and that in all the cases the termination
condition must be reached. If a recursion is too deep or infinite there will be
a stack overlow and the program will terminate. First, I will show you an
iterative version with a function. Let us try to compute factorial of an
integer:

\begin{minted}[frame=single]{c}
#include <stdio.h>

long long fact(int input);

int main()
{
  int input=0;

  printf("Enter a number whose input has to be computed:\n");
  scanf("%d", &input);

  printf("Factorial of %d is %lld.\n", input, fact(input));

  return 0;
}

long long fact(int input)
{
  long long output=1;
  while(input!=0)
  {
    output*=input;
    input--;
  }

  return output;
}
\end{minted}
and the output is:
\\\\\texttt{Enter a number whose factorial has to be computed:\\
17\\
Factorial of 17 is 355687428096000.\\\\}
Now we will see the recursive version:

\begin{minted}[frame=single]{c}
#include <stdio.h>

long long fact(int input);

int main()
{
  int input=0;

  printf("Enter a number whose factorial has to be computed:\n");
  scanf("%d", &input);

  printf("Factorial of %d is %lld.\n", input, fact(input));

  return 0;
}

long long fact(int input)
{
  if(input==0)
    return 1;
  else
    return fact(input-1)*input;
}
\end{minted}
and the output is:
\\\\\texttt{Enter a number whose factorial has to be computed:\\
16\\
Factorial of 16 is 20922789888000.\\\\}
Recursion is very simple yet may be very deceptive to understand for
beginners. Let us dissect the code. Our input was 16 so \texttt{input} will not
match and \texttt{return fact(15)*16;} will be executed. Here, before
\texttt{fact(16)} can return \texttt{fact(15)} has to return. And, similarly
before \texttt{fact(15)} can return \texttt{fact(14)} has to return. Now, note
that for \texttt{fact(0)} there is no such condition and it can return 1 making
it possible for \texttt{fact(1)} to return, which, in turn will make it posible
for \texttt{fact(2)} to return and so on. So, what is happening is function is
calling itself by creating more and more function frames and when the
termination condition reaches the stack unwinds.

Let us consider one more famous example for recursive function, that is of
computing Fibonacci numbers. The Fibonacci series is given by:

$$F_n = F_{n-1} + F_{n-2}$$

where first two numebrs are given by:

$$ F_0 = 0 \text{~and~} F_1 = 1$$ or
$$ F_0 = 1 \text{~and~} F_1 = 1$$

First, consider the iterative version:
